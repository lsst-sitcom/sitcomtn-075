\begin{table}[H]
    \footnotesize
    \begin{tabular}{ p{0.3\linewidth}  p{0.7\linewidth} }
    \toprule
    \textbf{Attribute} & \textbf{Notes} \\
    \midrule
    Primary Objectives & \tabitem AOS commissioning is driving -- use the subset of in-focus images \\
      & \tabitem Re-verification of infrastructure to collect and transfer data, process with Science Pipelines, produce and visualize QA metrics and plots \\
      & \tabitem First LSSTCam on-sky tests of finding reference stars for astrometric (wcs) and photometric (zeropoint) solutions, PSF determination \\
      & \tabitem Initial re-verification of ISR / calibration products with in-focus images \\
      & \tabitem Initial testing of offsets between amplifiers \\
      & \tabitem Initial studies of delivered image quality \\
    Additional Goals & \tabitem Initial check of optical system throughput (zeropoint) relative to expectation \\
    \midrule
    Prerequisites & \tabitem LED flats \\
    \midrule
    Total Visits & Driven by AOS commissioning needs. Expect several thousand (partially) in-focus visits over a period of several weeks. \\
    \midrule
    Overlapping Visits & Driven by AOS commissioning needs. For many tests, expect visits to be scattered across sky and that analysis will focus on individual visits at this stage. \\
    \midrule
    Band Coverage & Driven by AOS commissioning needs. Expect observations in multiple bands. \\
    \midrule
    Area & Driven by AOS commissioning needs. In general, expecting visits to be scattered across sky and that analysis will focus on individual visits at this stage. \\
    \midrule
    Pointing / Sky Region & Pointings likely to be distributed widely across the sky; might span range of stellar densities. \\
    \midrule
    Visit / Exposure Time & Driven by AOS commissioning needs. Potentially a range of exposure times. \\
    \midrule
    Dither Pattern & Driven by AOS commissioning needs. Conservatively plan that same fields will not be repeated. \\
    \midrule
    Cadence & Driven by AOS commissioning needs. Conservatively plan that same fields will not be repeated. \\
    \midrule
    Time Baseline & Driven by AOS commissioning needs. Conservatively plan that same fields will not be repeated. Most likely an expanding set of individual visits taken across several weeks. \\
    \midrule
    Delivered Image Quality & Heterogeneous delivered image quality. Anticipate gradients in delivered image quality across the FoV. \\
      & \tabitem 1.0 delivered PSF FWHM across the full FoV \\
    \midrule
    Airmass distribution & Driven by AOS commissioning needs. Anticipate scans in elevation during LUT verification. \\
    \midrule
    Environmental Conditions & Intentionally spanning range of temperatures, wind velocities and directions, atmospheric conditions. \\
    \midrule
    Representative Dataset Types & DRP mode: \\
      & \tabitem Source \\
      \midrule
    Success Criteria & \tabitem Initial alignment and focus of optics and camera \\
     & \tabitem Obtain astrometric, PSF, and zeropoint solution for at least a fraction of sensors on the focal plane \\
     & \tabitem Verify optical system throughput in central region of FoV (e.g., using zeropoint) \\
    \bottomrule
    \end{tabular}
    \caption{AOS Commissioning In-focus}
  \end{table}