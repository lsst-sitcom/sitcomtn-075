\documentclass[SE,authoryear,toc,lsstdraft]{lsstdoc}
\input{meta}

% Package imports go here.

% Local commands go here.

%If you want glossaries
%\input{aglossary.tex}
%\makeglossaries

\title{System On-sky Test Plan}

% Optional subtitle
% \setDocSubtitle{A subtitle}

\author{%
System On-sky Test Plan Working Group: Yusra AlSayyad, Keith Bechtol, Erik Dennihy, Patrick Ingraham, Scot Kleinman, Robert Lupton, Tiago Ribeiro, Eli Rykoff, Sandrine Thomas
}

\setDocRef{SITCOMTN-075}
\setDocUpstreamLocation{\url{https://github.com/lsst-sitcom/sitcomtn-075}}

\date{\vcsDate}

% Optional: name of the document's curator
% \setDocCurator{The Curator of this Document}

\setDocAbstract{%
Plan for the sequence of on-sky and in-dome data acquisition, data processing, and verification activities during the on-sky commissioning period with LSSTCam, considering AOS commissioning, calibration, and Science Programs together.
}

% Change history defined here.
% Order: oldest first.
% Fields: VERSION, DATE, DESCRIPTION, OWNER NAME.
% See LPM-51 for version number policy.
\setDocChangeRecord{%
  \addtohist{1}{YYYY-MM-DD}{Unreleased.}{Keith Bechtol}
}

% Tables
\usepackage{booktabs}
\usepackage{array}
\newcommand{\tabitem}{~~\llap{\textbullet}~~}

% Units
\newcommand{\unit}[1]{\ensuremath{\mathrm{\,#1}}\xspace}
\newcommand{\visits}{\unit{visits}}
\newcommand{\bands}{\unit{bands}}
\newcommand{\pointings}{\unit{pointings}}
\newcommand{\epochs}{\unit{epochs}}
\newcommand{\degree}{\unit{deg}}

\begin{document}

% Create the title page.
\maketitle
% Frequently for a technote we do not want a title page  uncomment this to remove the title page and changelog.
% use \mkshorttitle to remove the extra pages

% ADD CONTENT HERE
% You can also use the \input command to include several content files.

\section{Introduction}

This work planning document supplements the Rubin Observatory Commissioning Plan \citedsp{LSE-79}, providing a more detailed narrative description for the sequence of on-sky and in-dome data acquisition, data processing, and verification campaigns during the period of LSSTCam on-sky commissioning, considering Active Optics System (AOS) commissioning, calibration systems, and Science Programs together.
The objective is to provide sufficient detail to understand linkages between activities in the overall commissioning schedule and to empower the team to progress on specific work packages for each of the activities described, e.g., write observing scripts / scheduler configurations, plan data processing campaigns, develop science verification and validation analyses.
We identify open questions for further study.
Detailed work planning will be captured in other tools (e.g., P6 and Jira).

\section{Sequence of on-sky commissioning activities}

\section{Calibration Datasets}

% See https://jira.lsstcorp.org/browse/DM-39997
% SITCOMTN-086 Rubin Baseline Calibration Plan https://sitcomtn-086.lsst.io/
% SITCOMTN-087 Calibration System Milestone Summary https://sitcomtn-087.lsst.io/


\section{On-sky Example Datasets}

\subsection{AOS Commissioning In-Focus}

\begin{table}[H]
    \footnotesize
    \begin{tabular}{ p{0.3\linewidth}  p{0.7\linewidth} }
    \toprule
    \textbf{Attribute} & \textbf{Notes} \\
    \midrule
    Primary Objectives & \tabitem OS commissioning is driving – use the subset of in-focus images \\
      & \tabitem Re-verification of infrastructure to collect and transfer data, process with Science Pipelines, produce and visualize QA metrics and plots \\
      & \tabitem First LSSTCam on-sky tests of finding reference stars for astrometric (wcs) and photometric (zeropoint) solutions, PSF determination \\
      & \tabitem Initial re-verification of ISR / calibration products with in-focus images \\
      & \tabitem Initial testing of offsets between amplifiers \\
      & \tabitem Initial studies of delivered image quality \\
    Additional Goals & \tabitem Initial check of optical system throughput (zeropoint) relative to expectation \\
    \midrule
    Prerequisites & \tabitem LED flats \\
    \midrule
    Total Visits & Driven by AOS commissioning needs. Expect several thousand (partially) in-focus visits over a period of several weeks. \\
    \midrule
    Overlapping Visits & Driven by AOS commissioning needs. For many tests, expect visits to be scattered across sky and that analysis will focus on individual visits at this stage. \\
    \midrule
    Band Coverage & Driven by AOS commissioning needs. Expect observations in multiple bands. \\
    \midrule
    Area & Driven by AOS commissioning needs. In general, expecting visits to be scattered across sky and that analysis will focus on individual visits at this stage. \\
    \midrule
    Pointing / Sky Region & Pointings likely to be distributed widely across the sky \\
    \midrule
    Visit / Exposure Time & Driven by AOS commissioning needs. Potentially a range of exposure times. \\
    \midrule
    Dither Pattern & Driven by AOS commissioning needs. Conservatively plan that same fields will not be repeated. \\
    \midrule
    Cadence & Driven by AOS commissioning needs. Conservatively plan that same fields will not be repeated. \\
    \midrule
    Time Baseline & Driven by AOS commissioning needs. Conservatively plan that same fields will not be repeated. Most likely an expanding set of individual visits taken across several weeks. \\
    \midrule
    Delivered Image Quality & Heterogeneous delivered image quality. Anticipate gradients in delivered image quality across the FoV. \\
      & \tabitem 1.0 delivered PSF FWHM across the full FoV \\
    \midrule
    Airmass distribution & Driven by AOS commissioning needs. Anticipate scans in elevation during LUT verification. \\
    \midrule
    Environmental Conditions & Intentionally spanning range of temperatures, wind velocities and directions, atmospheric conditions. \\
    \midrule
    Representative Dataset Types & DRP mode: \\
      & \tabitem Source \\
      \midrule
    Success Criteria & \tabitem Initial alignment and focus of optics and camera \\
     & \tabitem Obtain astrometric, PSF, and zeropoint solution for at least a fraction of sensors on the focal plane \\
     & \tabitem Verify optical system throughput in central region of FoV (e.g., using zeropoint) \\
    \bottomrule
    \end{tabular}
    \caption{AOS Commissioning In-focus}
  \end{table}

\subsection{Exposure Time Scans}

\begin{table}[H]
    \footnotesize
    \begin{tabular}{ p{0.3\linewidth}  p{0.7\linewidth} }
    \toprule
    \textbf{Attribute} & \textbf{Notes} \\
    \midrule
    Primary Objectives & \tabitem Determine reference flat (illumination correction) independent of CBP \\
      & \tabitem Determine instrumental astrometric model \\
    \midrule
    Additional Goals & \tabitem Initial testing of building coaddition and difference imaging \\
    \midrule
    Prerequisites & \tabitem White light flats have been acquired \\
      & \tabitem Monochromatic flats have been acquired \\
    \midrule
    Total Visits & 2 visits $\times$ 20 exposure times $\times$ 6 bands = 240 visits (anticipate roughly 1 hr per band) \\
    \midrule
    Overlapping Visits & Single pointing, minimal dither \\
    \midrule
    Band Coverage & $ugrizy$ (must include $u$) \\
    \midrule
    Area & $\sim10$ deg$^2$ \\
    \midrule
    Pointing / Sky Region & Avoid regions with bright sources  \\
    \midrule
    Visit / Exposure Time & Series of visits with exposure times ranging from 1 second to 300 seconds \\
    \midrule
    Dither Pattern & Single pointing, minimal dither \\
    \midrule
    Cadence & Take repeated observation with same band consecutively. \\
    \midrule
    Time Baseline & Sequential exposures \\
    \midrule
    Delivered Image Quality & Could begin with 2.0 arcsec delivered PSF FWHM across the full FoV \\
    \midrule
    Airmass distribution & $z < 1.2$ \\
    \midrule
    Environmental Conditions & \tabitem Photometric \\
    \midrule
    Representative Dataset Types & DRP mode: \\
      & \tabitem Source \\
      \midrule
    Success Criteria & \\
    \bottomrule
    \end{tabular}
    \caption{Exposure Time Scans}
  \end{table}

\subsection{Bright Star Scans}

\begin{table}[H]
    \footnotesize
    \begin{tabular}{ p{0.3\linewidth}  p{0.7\linewidth} }
    \toprule
    \textbf{Attribute} & \textbf{Notes} \\
    \midrule
    Primary Objectives & \tabitem Characterize ghosts and scattered light \\
      & \tabitem Bright star profile / wings of the PSF \\
      & \tabitem Robustness of single-frame processing to presence of bright star \\
      & \tabitem Appropriate mask design for bright stars \\
      & \tabitem Test ISR including, crosstalk, bleed trails \\
    \midrule
    Additional Goals &  \\
    \midrule
    Prerequisites & \tabitem LED flats have been acquired \\
      & \tabitem Monochromatic flats have been acquired \\
    \midrule
    Total Visits & 4 fields \times 30 visits \times 6 bands = 720 visits (scheduled with observations in 3 bands at a time?) \\
    \midrule
    Overlapping Visits & $\sim30$ partially overlapping \\
    \midrule
    Band Coverage & $(u)griz(y)$, prioritize $griz$ initially \\
    \midrule
    Area & $\sim10$ deg$^2$ per field \\
    \midrule
    Pointing / Sky Region & Fields containing a star bright enough to compromise measurements on: \\
      & \tabitem Amplifier \\
      & \tabitem Detector \\
      & \tabitem Raft \\
      & \tabitem Significant fraction of focal plane \\
    \midrule
    Visit / Exposure Time & Standard visit \\
    \midrule
    Dither Pattern & Scan pointing across bright star so that the object appears at multiple locations on the focal plane as well as slightly outside the field of view.  \\
    \midrule
    Cadence & Not constrained, probably take observations of a given field together in a single epoch \\
    \midrule
    Time Baseline & Not constrained, probably take observations of a given field together in a single epoch \\
    \midrule
    Delivered Image Quality & Could begin with 2.0 arcsec delivered PSF FWHM across the full FoV \\
    \midrule
    Airmass distribution & Not constrained, $z < 1.4$ \\
    \midrule
    Environmental Conditions & \tabitem Photometric \\
    \midrule
    Representative Dataset Types & \tabitem DRP mode: Source, Object \\
      & Need to build coadd images to inspect residual effects of scattered light and ghosts \\
      \midrule
    Success Criteria & \\
    \bottomrule
    \end{tabular}
    \caption{Exposure Time Scans}
  \end{table}

\subsection{Dense Dithered Star Field}

\begin{table}[H]
    \footnotesize
    \begin{tabular}{ p{0.3\linewidth}  p{0.7\linewidth} }
    \toprule
    \textbf{Attribute} & \textbf{Notes} \\
    \midrule
    Primary Objectives & \tabitem Determine reference flat (illumination correction) independent of CBP \\
      & \tabitem Determine instrumental astrometric model \\
    \midrule
    Additional Goals & \tabitem Initial testing of building coaddition and difference imaging \\
    \midrule
    Prerequisites & \tabitem LED flats have been acquired \\
      & \tabitem Monochromatic flats have been acquired (iterations approaching System First Light) \\
    \midrule
    Total Visits & 120 \visits $\times$ 6 \epochs = 720 \visits (8 hrs total; 6 epochs of 80 min each) \\
    \midrule
    Overlapping Visits & 30 visits in each of ugri (dark time) or rizy (bright time) in each epoch; epochs span range of airmass; comparable to single year of LSST WFD integrated exposure \\
    \midrule
    Band Coverage & ugrizy (4 bands in a given epoch is sufficient) \\
    \midrule
    Area & Central region of $\sim$10 deg$^2$ (focal plane dithers around a single central pointing) \\
    \midrule
    Pointing / Sky Region & Region of moderately high stellar density, but not so dense that blending is a concern. Low interstellar extinction \\
    \midrule
    Visit / Exposure Time & Standard visit \\
    \midrule
    Dither Pattern & Dithers from sensor scale up to focal plane scale with translation (rotation needed??) around a central pointing so that the same stars appear on many different sensors across the focal plane \\
    \midrule
    Cadence & Repeated visits within a single epoch. Alternate between filters within an epoch so that stars are observed in griz bands within a single epoch. \\
    \midrule
    Time Baseline & $\sim$80 minutes within epoch; epochs can be on separate nights (close spacing preferred) \\
    \midrule
    Delivered Image Quality & Can begin when no strong gradients in PSF across FoV \\
      & \tabitem 0.7 arcsec system contribution to PSF FWHM across the full FoV \\
      & \tabitem 1.0 arcsec delivered PSF FWHM across the full FoV \\
    \midrule
    Airmass distribution & Observe same field in in three epochs to sample a range of airmass: 1.0, 1.4, 2.0 \\
    \midrule
    Environmental Conditions & \tabitem Photometric \\
    \midrule
    Representative Dataset Types & DRP mode: \\
      & \tabitem Source, Object, ForcedSource, DiaSource, DiaObject, DiaForcedSource \\
      & AP mode: (could be replay) \\
      & \tabitem DiaObject, DiaSource, DIAForcedSource \\
      \midrule
    Success Criteria & \tabitem Verification of astrometric and photometric internal solution \\
      & \tabitem Verification of acceptable image quality across full FoV across range of airmass as part of System First Light milestone \\
    \bottomrule
    \end{tabular}
    \caption{Dense Dithered Star Field}
  \end{table}

\subsection{Deep Drilling Field (DDF)}

\begin{table}[H]
    \footnotesize
    \begin{tabular}{ p{0.3\linewidth}  p{0.7\linewidth} }
    \toprule
    \textbf{Attribute} & \textbf{Notes} \\
    \midrule
    Primary Objectives & \tabitem Verify that scattered light and other artifacts can be mitigated with nominal LSST DDF dither strategy and acceptable masking \\
    \midrule
    Additional Goals & \tabitem Test building coadds $>100$ visits \\
      & \tabitem Test DIA \\
    \midrule
    Prerequisites & \tabitem Acceptable image quality across full FoV at a particular airmass \\
      & \tabitem Initial analysis of Bright Star Scans to evaluate ghosts and scattered light in LSSTCam \\
    \midrule
    Total Visits & 5 \epochs $\times$ 40 \visits $\times$ 6 bands = 1200 \visits (6 epochs w/ 200 visits per epoch given 5 filters in LSSTCam at a time) \\
    \midrule
    Overlapping Visits & $ugrizy$ = (200, 200, 200, 200, 200, 200) \\
    \midrule
    Band Coverage & $ugrizy$ (aim for 6 band coverage) \\
    \midrule
    Area & $\sim$10 deg$^2$ (single DDF) \\
    \midrule
    Pointing / Sky Region & One of the LSST Deep Drilling Fields \\
    \midrule
    Visit / Exposure Time & Standard visit \\
    \midrule
    Dither Pattern & Nominal LSST DDF dithering strategy. Note that will probably need to observe with a variety of transational offsets and rotator angles to simulate the distribution that would be obtained in LSST survey. \\
    \midrule
    Cadence & Repeated consecutive visits within an epoch. Alternate between bands. \\
    \midrule
    Time Baseline & Epochs on separate nights \\
    \midrule
    Delivered Image Quality & \tabitem 0.7 system contribution to PSF FWHM across the full FoV \\
      & \tabitem 1.0 delivered PSF FWHM across the full FoV \\
    \midrule
    Airmass distribution & Sample nominal airmass range from LSST 10-year survey \\
    \midrule
    Environmental Conditions & Need dark time for $ugr$ \\
    \midrule
    Representative Dataset Types & DRP mode: \\
      & \tabitem Source \\
      & \tabitem Object \\
      & \tabitem ForcedSource \\
      & \tabitem DiaSource \\
      & \tabitem DiaObject \\
      & \tabitem DiaForcedSource \\
      & AP mode: \\
      & \tabitem DiaObject \\
      & \tabitem DiaSource \\
      & \tabitem DIAForcedSource \\
      \midrule
    Success Criteria & Technical report to inform recommendation for initial DDF dither strategy at the start of LSST 10-year survey \\
    \bottomrule
    \end{tabular}
    \caption{Deep Drilling Field}
  \end{table}

\subsection{Crowded Fields}

%Questions:
%\begin{itemize}
%  \item Band coverage: is $u$ band critical?
%\end{itemize}

%\begin{table}[H]
    \footnotesize
    \begin{tabular}{ p{0.3\linewidth}  p{0.7\linewidth} }
    \toprule
    \textbf{Attribute} & \textbf{Notes} \\
    \midrule
    Primary Objectives & \tabitem Evaluate scientific performance of DIA in a high density stellar field with enough repeated visits and dithers to be representative of LSST WFD over the Galactic plane \\
    \midrule
    Additional Goals & \tabitem Characterize deblending for deep coadds in crowded fields \\
      & \tabitem Characterize interstellar extinction effects  \\
    \midrule
    Prerequisites & \tabitem Acceptable image quality across full FoV at a particular airmass \\
      & \tabitem Analysis of Bright Star Scans \\
    \midrule
    Total Visits & \tabitem Total: 4 pointings $\times$ 15 \visits per band $\times$ 5 bands $\times$ 4 \epochs = 1200 \visits (16 total epochs of 75 visits each) \\
      & \tabitem Prioritize completion of epochs for a given pointing \\
    \midrule
    Overlapping Visits & $\sim60$ visits per band in central region (less on periphery due to dithering) \\
    \midrule
    Band Coverage & $(u)griz(y)$, priorize $u$ or $y$ depending on filter set and field \\
    \midrule
    Area & $\sim$10 deg$^2$ central region per field \\
    \midrule
    Pointing / Sky Region & High stellar density regions (e.g., Galactic plane, LMC/SMC, globular cluster, resolved galaxy) sampling a range of stellar densities. \\
    \midrule
    Visit / Exposure Time & Standard visit \\
    \midrule
    Dither Pattern & Nominal LSST WFD dithering strategy. \\
    \midrule
    Cadence & Repeated consecutive visits within an epoch.  \\
    \midrule
    Time Baseline & Epochs on different nights. \\
    \midrule
    Delivered Image Quality & \tabitem 0.7 system contribution to PSF FWHM across the full FoV \\
      & \tabitem 1.0 delivered PSF FWHM across the full FoV \\
    \midrule
    Airmass distribution & Sample airmass from $1.0 < z < 1.4$ \\
    \midrule
    Environmental Conditions &  \\
    \midrule
    Representative Dataset Types & DRP mode: \\
      & \tabitem Source, Object, ForcedSource, DiaSource, DiaObject, DiaForcedSource \\
      & AP mode (could be replay): \\
      & \tabitem DiaObject, DiaSource, DIAForcedSource \\
      \midrule
    Success Criteria & Science validation report of DIA performance in crowded stellar fields \\
    \bottomrule
    \end{tabular}
    \caption{Crowded Fields}
  \end{table}

\subsection{Science Validation Surveys}

Plan to interleave survey components so that both survey components have observations on many distinct nights

Option to emphasize deep component in case that there is processing time needed for template generation and/or infrastructure issues for AP that need to be worked

There is probably some careful thinking to do with the optimization of the scheduler on whether to prioritize template building initially, or to prioritize getting some repeated visits in areas with templates to get an early look at DIA. I think based on discussions with AP team that prioritization of template building is preferred at the start. If there is uncertainty on the duration of the SV survey period, there will also be thinking to do on balance of adding area versus adding more repeated observations of fields with existing templates.

\begin{table}[H]
    \footnotesize
    \begin{tabular}{ p{0.3\linewidth}  p{0.7\linewidth} }
    \toprule
    \textbf{Attribute} & \textbf{Notes} \\
    \midrule
    Primary Objectives & \tabitem Deep survey component is optimized for testing coadds at LSST 10-year survey full depth and beyond \\
      & \tabitem Full rehearsal of nighttime and daytime operational procedures \\
    \midrule
    Additional Goals & \tabitem Dataset of high legacy value to inform ongoing Science Pipeline development and enable broad set of science validation investigations \\
    \midrule
    Prerequisites & \tabitem System First Light \\
    \midrule
    Total Visits & \tabitem First 30 days: $\sim11$ pointings $\times$ 825 visits per pointing = 9075 visits ($\sim15$ night equivalents) \\
    \midrule
    Overlapping Visits & \tabitem First 30 days: 825 visits at each pointing $(u, g, r, i, z, y) = (56, 80, 184, 184, 160, 160)$ (nominal LSST WFD 10-year integrated exposure) \\
      & \tabitem Extension: increase depth to 1650 visits at each pointing $(u, g, r, i, z, y) = (112, 160, 368, 368, 320, 320)$ (equivalent to LSST WFD 20-year integrated exposure) \\
    \midrule
    Band Coverage & $ugrizy$ \\
    \midrule
    Area & \tabitem First 30 days: $\sim1100$ deg$^2$ \\
    \midrule
    Pointing / Sky Region & Single contiguous region overlapping one of the LSST DDF \\
    \midrule
    Visit / Exposure Time & Standard visit \\
    \midrule
    Dither Pattern & \tabitem Nominal LSST WFD dither pattern \\
      & \tabitem Note that will probably need to observe the deep region with a variety of translational offsets and rotator angles to simulate the distribution that would be obtained in LSST survey. \\
    \midrule
    Cadence & Steadily build integrated exposure across the survey. \\
      & During a 30 day period, each pointing would receive average of ~25 visits on each night. Due to lunar cycle, visits in reddest and bluest bands are likely to be more concentrated during bright and dark time, respectively. \\
    \midrule
    Time Baseline & Minimum 30 days. \\
    \midrule
    Delivered Image Quality & Expect a range similar to distribution in 10-year LSST survey as we sample a range of observing conditions. Anticipate ongoing system optimization. \\
    \midrule
    Airmass distribution & Observe same field in in three epochs to sample a range of airmass of 1.0, 1.4, 2.0 \\
    \midrule
    Environmental Conditions & Sample range of environmental conditions over a period of at least 30 days, including sky brightness (lunar cycle), seeing, atmosphere transparency, wind speed and direction, temperature \\
    \midrule
    Representative Dataset Types & DRP mode: \\
      & \tabitem Source \\
      & \tabitem Object \\
      & \tabitem ForcedSource \\
      & \tabitem DiaSource \\
      & \tabitem DiaObject \\
      & \tabitem DiaForcedSource \\
      & AP mode: \\
      & \tabitem DiaObject \\
      & \tabitem DiaSource \\
      & \tabitem DIAForcedSource \\
      \midrule
    Success Criteria & Collect data to verify full set of system-level science performance requirements at survey scale, with emphasis on characterizing systematics for deep coadds \\
    \bottomrule
    \end{tabular}
    \caption{Science Validation Survey Deep component}
  \end{table}

\begin{table}[H]
    \footnotesize
    \begin{tabular}{ p{0.3\linewidth}  p{0.7\linewidth} }
    \toprule
    \textbf{Attribute} & \textbf{Notes} \\
    \midrule
    Primary Objectives & \tabitem Wide survey component is optimized for testing Alert Prodction at survey scale, including infrastructure and science validation \\
      & \tabitem Full rehearsal of nighttime and daytime operational procedures \\
    \midrule
    Additional Goals & \tabitem Science validation of Data Release Processing over wide area \\
      & \tabitem Produce templates over wide area to enhance Alert Production opportunities during first year of LSST \\
    \midrule
    Prerequisites & \tabitem System First Light \\
    \midrule
    Total Visits & \tabitem First 30 nights: $\sim110$ pointings $\times$ 80 visits per pointing = 9000 \visits ($\sim15$ night equivalents) \\
    \midrule
    Overlapping Visits & 80 visits at each pointing $(g, r, i, z) = (20, 20, 20, 20)$ (comparable to single year of LSST WFD integrated exposure) \\
    \midrule
    Band Coverage & griz \\
    \midrule
    Area & \tabitem First 30 nights: $\sim1100$ deg$^2$ \\
      & \tabitem Extension: prioritize increased area coverage, attempting to maintain depth uniformity \\
    \midrule
    Pointing / Sky Region & \tabitem One or more large large contiguous regions placed to optimize scheduling flexibility, considering the SV survey Deep region. \\
      & \tabitem Span a range of stellar density. \\
      & \tabitem Cross ecliptic. \\
      & \tabitem Overlap with external reference datasets to the extent possible. \\
    \midrule
    Visit / Exposure Time & Standard visit \\
    \midrule
    Dither Pattern & Nominal LSST WFD dither pattern \\
    \midrule
    Cadence & \tabitem Initial emphasis on building up template coverage; then repeated observations for testing DIA. \\
      & \tabitem Include Solar System cadence (pairs of visits separated by 30 min in different filters). \\
      & \tabitem Steadily build integrated exposure across the survey, sample of range of cadences for repeated observations of same field. \\
    \midrule
    Time Baseline & Minimum 30 days. \\
    \midrule
    Delivered Image Quality & Expect a range similar to distribution in 10-year LSST survey as we sample a range of observing conditions. Anticipate ongoing system optimization. \\
    \midrule
    Airmass distribution & Span range of airmass from $1.0 < z < 2.0$, concentrated in typical airmass range expected for LSST survey $1.0 < z < 1.4$. \\
    \midrule
    Environmental Conditions & Sample range of environmental conditions over a period of at least 30 days, including sky brightness (lunar cycle), seeing, atmosphere transparency, wind speed and direction, temperature \\
    \midrule
    Representative Dataset Types & DRP mode: \\
      & \tabitem Source, Object, ForcedSource, DiaSource, DiaObject, DiaForcedSource \\
      & AP mode: \\
      & \tabitem DiaObject, DiaSource, DIAForcedSource \\
      \midrule
    Success Criteria & Collect data to verify full set of system-level science performance requirements at survey scale, with emphasis on Alert Production \\
    \bottomrule
    \end{tabular}
    \caption{Science Validation Survey Wide component}
  \end{table}

\section{Mapping to System-level Science Performance Requirements}

\appendix
% Include all the relevant bib files.
% https://lsst-texmf.lsst.io/lsstdoc.html#bibliographies
\section{References} \label{sec:bib}
\renewcommand{\refname}{} % Suppress default Bibliography section
\bibliography{local,lsst,lsst-dm,refs_ads,refs,books}

% Make sure lsst-texmf/bin/generateAcronyms.py is in your path
\section{Acronyms} \label{sec:acronyms}
\addtocounter{table}{-1}
\begin{longtable}{p{0.145\textwidth}p{0.8\textwidth}}\hline
\textbf{Acronym} & \textbf{Description}  \\\hline

AOS & Active Optics System \\\hline
AP & Alert Production \\\hline
ASAP & As Soon As Possible \\\hline
CBP & Collimated Beam Projector \\\hline
CM & Configuration Management \\\hline
DDF & Deep Drilling Fields \\\hline
DIA & Difference Image Analysis \\\hline
DM & Data Management \\\hline
DRP & Data Release Production \\\hline
FWHM & Full Width at Half-Maximum \\\hline
FoV & Field of View (also denoted FOV) \\\hline
HSC & Hyper Suprime-Cam \\\hline
ISR & Instrument Signal Removal \\\hline
LED & Light-Emitting Diode \\\hline
LMC & Large Magellanic Cloud \\\hline
LSE & LSST Systems Engineering (Document Handle) \\\hline
LSST & Legacy Survey of Space and Time (formerly Large Synoptic Survey Telescope) \\\hline
LUT & Look-Up Table \\\hline
M1M3 & Primary Mirror Tertiary Mirror \\\hline
ORR & Operations Readiness Review \\\hline
OS & Operating System \\\hline
PSF & Point Spread Function \\\hline
QA & Quality Assurance \\\hline
SE & System Engineering \\\hline
SITCOM & System Integration, Test and Commissioning \\\hline
SMC & Small Magellanic Cloud \\\hline
SV & Science Validation \\\hline
TBD & To Be Defined (Determined) \\\hline
USDF & United States Data Facility \\\hline
WFD & Wide Fast Deep \\\hline
arcsec & arcsecond second of arc (unit of angle) \\\hline
deg & degree; unit of angle \\\hline
\end{longtable}

% If you want glossary uncomment below -- comment out the two lines above
%\printglossaries





\end{document}
