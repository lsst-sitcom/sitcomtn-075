\documentclass[SE,authoryear,toc]{lsstdoc}
\input{meta}

% Package imports go here.

% Local commands go here.

%If you want glossaries
%\input{aglossary.tex}
%\makeglossaries

\title{System On-sky Test Plan}

% Optional subtitle
% \setDocSubtitle{A subtitle}

\author{%
Keith Bechtol
}

\setDocRef{SITCOMTN-075}
\setDocUpstreamLocation{\url{https://github.com/lsst-sitcom/sitcomtn-075}}

\date{\vcsDate}

% Optional: name of the document's curator
% \setDocCurator{The Curator of this Document}

\setDocAbstract{%
Plan for the sequence of on-sky and in-dome data acquisition, data processing, and verification activities during the on-sky commissioning period with LSSTCam, considering AOS commissioning, calibration, and Science Programs together.
}

% Change history defined here.
% Order: oldest first.
% Fields: VERSION, DATE, DESCRIPTION, OWNER NAME.
% See LPM-51 for version number policy.
\setDocChangeRecord{%
  \addtohist{1}{YYYY-MM-DD}{Unreleased.}{Keith Bechtol}
}


\begin{document}

% Create the title page.
\maketitle
% Frequently for a technote we do not want a title page  uncomment this to remove the title page and changelog.
% use \mkshorttitle to remove the extra pages

% ADD CONTENT HERE
% You can also use the \input command to include several content files.

\appendix
% Include all the relevant bib files.
% https://lsst-texmf.lsst.io/lsstdoc.html#bibliographies
\section{References} \label{sec:bib}
\renewcommand{\refname}{} % Suppress default Bibliography section
\bibliography{local,lsst,lsst-dm,refs_ads,refs,books}

% Make sure lsst-texmf/bin/generateAcronyms.py is in your path
\section{Acronyms} \label{sec:acronyms}
\addtocounter{table}{-1}
\begin{longtable}{p{0.145\textwidth}p{0.8\textwidth}}\hline
\textbf{Acronym} & \textbf{Description}  \\\hline

AOS & Active Optics System \\\hline
AP & Alert Production \\\hline
ASAP & As Soon As Possible \\\hline
CBP & Collimated Beam Projector \\\hline
CM & Configuration Management \\\hline
DDF & Deep Drilling Fields \\\hline
DIA & Difference Image Analysis \\\hline
DM & Data Management \\\hline
DRP & Data Release Production \\\hline
FWHM & Full Width at Half-Maximum \\\hline
FoV & Field of View (also denoted FOV) \\\hline
HSC & Hyper Suprime-Cam \\\hline
ISR & Instrument Signal Removal \\\hline
LED & Light-Emitting Diode \\\hline
LMC & Large Magellanic Cloud \\\hline
LSE & LSST Systems Engineering (Document Handle) \\\hline
LSST & Legacy Survey of Space and Time (formerly Large Synoptic Survey Telescope) \\\hline
LUT & Look-Up Table \\\hline
M1M3 & Primary Mirror Tertiary Mirror \\\hline
ORR & Operations Readiness Review \\\hline
OS & Operating System \\\hline
PSF & Point Spread Function \\\hline
QA & Quality Assurance \\\hline
SE & System Engineering \\\hline
SITCOM & System Integration, Test and Commissioning \\\hline
SMC & Small Magellanic Cloud \\\hline
SV & Science Validation \\\hline
TBD & To Be Defined (Determined) \\\hline
USDF & United States Data Facility \\\hline
WFD & Wide Fast Deep \\\hline
arcsec & arcsecond second of arc (unit of angle) \\\hline
deg & degree; unit of angle \\\hline
\end{longtable}

% If you want glossary uncomment below -- comment out the two lines above
%\printglossaries





\end{document}
