\begin{table}[H]
    \footnotesize
    \begin{tabular}{ p{0.3\linewidth}  p{0.7\linewidth} }
    \toprule
    \textbf{Attribute} & \textbf{Notes} \\
    \midrule
    Primary Objectives & \tabitem Determine reference flat (illumination correction) independent of CBP \\
      & \tabitem Determine instrumental astrometric model \\
    \midrule
    Additional Goals & \tabitem Initial testing of building coaddition and difference imaging \\
    \midrule
    Prerequisites & \tabitem White light flats have been acquired \\
      & \tabitem Monochromatic flats have been acquired \\
    \midrule
    Total Visits & 2 visits $\times$ 20 exposure times $\times$ 6 bands = 240 visits (anticipate roughly 1 hr per band) \\
    \midrule
    Overlapping Visits & Single pointing, minimal dither \\
    \midrule
    Band Coverage & $ugrizy$ (must include $u$) \\
    \midrule
    Area & $\sim10$ deg$^2$ \\
    \midrule
    Pointing / Sky Region & Avoid regions with bright sources  \\
    \midrule
    Visit / Exposure Time & Series of visits with exposure times ranging from 1 second to 300 seconds \\
    \midrule
    Dither Pattern & Single pointing, minimal dither \\
    \midrule
    Cadence & Take repeated observation with same band consecutively. \\
    \midrule
    Time Baseline & Sequential exposures \\
    \midrule
    Delivered Image Quality & Could begin with 2.0 arcsec delivered PSF FWHM across the full FoV \\
    \midrule
    Airmass distribution & $z < 1.2$ \\
    \midrule
    Environmental Conditions & \tabitem Photometric \\
    \midrule
    Representative Dataset Types & DRP mode: \\
      & \tabitem Source \\
      \midrule
    Success Criteria & \\
    \bottomrule
    \end{tabular}
    \caption{Exposure Time Scans}
  \end{table}