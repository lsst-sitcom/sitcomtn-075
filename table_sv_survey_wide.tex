\begin{table}[H]
    \footnotesize
    \begin{tabular}{ p{0.3\linewidth}  p{0.7\linewidth} }
    \toprule
    \textbf{Attribute} & \textbf{Notes} \\
    \midrule
    Primary Objectives & \tabitem Wide survey component is optimized for testing Alert Prodction at survey scale, including infrastructure and science validation \\
      & \tabitem Full rehearsal of nighttime and daytime operational procedures \\
    \midrule
    Additional Goals & \tabitem Science validation of Data Release Processing over wide area \\
      & \tabitem Produce templates over wide area to enhance Alert Production opportunities during first year of LSST \\
    \midrule
    Prerequisites & \tabitem System First Light \\
    \midrule
    Total Visits & \tabitem First 30 nights: $\sim110$ pointings $\times$ 80 visits per pointing = 9000 \visits ($\sim15$ night equivalents) \\
    \midrule
    Overlapping Visits & 80 visits at each pointing $(g, r, i, z) = (20, 20, 20, 20)$ (comparable to single year of LSST WFD integrated exposure) \\
    \midrule
    Band Coverage & griz \\
    \midrule
    Area & \tabitem First 30 nights: $\sim1100$ deg$^2$ \\
      & \tabitem Extension: prioritize increased area coverage, attempting to maintain depth uniformity \\
    \midrule
    Pointing / Sky Region & \tabitem One or more large large contiguous regions placed to optimize scheduling flexibility, considering the SV survey Deep region. \\
      & \tabitem Span a range of stellar density. \\
      & \tabitem Cross ecliptic. \\
      & \tabitem Overlap with external reference datasets to the extent possible. \\
    \midrule
    Visit / Exposure Time & Standard visit \\
    \midrule
    Dither Pattern & Nominal LSST WFD dither pattern \\
    \midrule
    Cadence & \tabitem Initial emphasis on building up template coverage; then repeated observations for testing DIA. \\
      & \tabitem Include Solar System cadence (pairs of visits separated by 30 min in different filters). \\
      & \tabitem Steadily build integrated exposure across the survey, sample of range of cadences for repeated observations of same field. \\
    \midrule
    Time Baseline & Minimum 30 days. \\
    \midrule
    Delivered Image Quality & Expect a range similar to distribution in 10-year LSST survey as we sample a range of observing conditions. Anticipate ongoing system optimization. \\
    \midrule
    Airmass distribution & Span range of airmass from $1.0 < z < 2.0$, concentrated in typical airmass range expected for LSST survey $1.0 < z < 1.4$. \\
    \midrule
    Environmental Conditions & Sample range of environmental conditions over a period of at least 30 days, including sky brightness (lunar cycle), seeing, atmosphere transparency, wind speed and direction, temperature \\
    \midrule
    Representative Dataset Types & DRP mode: \\
      & \tabitem Source, Object, ForcedSource, DiaSource, DiaObject, DiaForcedSource \\
      & AP mode: \\
      & \tabitem DiaObject, DiaSource, DIAForcedSource \\
      \midrule
    Success Criteria & Collect data to verify full set of system-level science performance requirements at survey scale, with emphasis on Alert Production \\
    \bottomrule
    \end{tabular}
    \caption{Science Validation Survey Wide component}
  \end{table}